\documentclass[11pt,a4paper,sans]{moderncv}
\moderncvstyle{classic} % Options : 'casual', 'classic', 'banking'
\moderncvcolor{orange}    % Options : 'blue', 'orange', 'green', 'red', 'purple'

\usepackage[utf8]{inputenc}
\usepackage[scale=0.75]{geometry}
\usepackage{graphicx}   % Requis pour inclure des images
\setlength{\hintscolumnwidth}{2.5cm}

% Informations personnelles
\name{Jérôme}{DESSEAUX}
\title{\textnormal{\textls[150]{TECHLEAD ENGINEER}}} % Titre (non-italique, plus d'espacement)
\address{29 allée du petit houx}{76230, Bois Guillaume, France}                             % Adresse (optionnel)
\phone[mobile]{(+33) (0)6 83 10 42 78}                           % Numéro de téléphone
\email{jerome.desseaux@skuld.fr}                         % Adresse email
\homepage{jeromedesseaux}                 % Site personnel (optionnel)
\social[github]{JeromeDesseaux}                % Profil GitHub (optionnel)
\quote{\textit{\textcolor{darkgray}{Concevoir des solutions innovantes. Maîtriser les coûts d'exploitation. Optimiser l'existant et fluidifier les processus de développement.}}} % Citation
% Ajouter votre photo
\photo[64pt][0.4pt]{moi.png} % Ajuster la taille si nécessaire

\begin{document}

% En-tête
\makecvtitle

% Résumé
\section{Résumé}
Ingénieur Tech Lead accompli avec plus de 8 ans d'expérience en leadership technique, développement full-stack, et traitement de données. Contribue à l'optimisation des projets existants 
tant sur le plan technique que sur le plan de l'organisation avec une réduction des coûts d'exploitation, une amélioration des performances, une réduction de la dette technique 
mais également un renforcement de l'équipe technique, de la qualité du code et des processus de développement.\\

\vspace{0.1cm}

Expert dans l'architecture de solutions évolutives et de la promotion de l'innovation grâce aux méthodologies agiles. Compétent dans une large gamme de langages de programmation avec une expertise plus prononcée sur Python et Javascript ainsi 
qu'une vaste expérience dans des frameworks tels que Django, React et Node.js. Capable de traduire des concepts techniques complexes en initiatives stratégiques et produits  
alignés avec les objectifs commerciaux. Engagé à favoriser une culture d'amélioration continue et de collaboration, 
en utilisant des technologies de pointe pour optimiser les performances, les coûts, fluidifier les échanges et obtenir des résultats concrets. 

% Compétences Techniques
\section{Compétences Techniques}
\subsection{Backend}
\cvitem{Langages}{Python, C\#, JavaScript, TypeScript, Java}
\cvitem{Frameworks}{Django, Flask, FastAPI, .NET, Spring Boot, Express.js, Nest.js}

\subsection{Frontend}
\cvitem{Langages}{JavaScript, TypeScript}
\cvitem{Frameworks}{React.js, Angular.js, Vue.js}

\subsection{Outils}
\cvitem{IDE}{Visual Studio Code, PyCharm, IntelliJ}
\cvitem{Versioning}{Git}
\cvitem{Gestion de Projet}{Trello, Jira}

\subsection{Systèmes d'Exploitation}
\cvitem{OS}{Windows, Linux, Mac}

\subsection{DevOps}
\cvitem{Cloud}{AWS, Azure}
\cvitem{Orchestration}{Kubernetes}
\cvitem{Conteneurisation}{Docker}

\subsection{Gestion de Projet}
\cvitem{Méthodologies}{Agile SCRUM}

% Expérience Professionnelle
\section{Expérience Professionnelle}

\cventry{2024}{Tech-Lead}{LENGOW}{Freelance}{Télétravail}{
    \begin{itemize}
        \item Mise en place d'un nouveau connecteur (Octopia) pour la plateforme Lengow, permettant l'intégration de nouvelles places de marché.
        \item Support de l'équipe dans le développement de nouvelles fonctionnalités et l'optimisation des fonctionnalités existantes.
        \item Assurer la qualité du code et le respect des meilleures pratiques.
        \item Participation aux choix techniques et à l'architecture de la plateforme.
        \item Assurer la surveillance technique de la plateforme et la résolution des incidents.
        \item Mise à jour des services actuels vers les dernières versions de Django et Python.
        \item Développement de l'utilisation de Docker comme outil de développement et de déploiement
        \begin{itemize}
            \item Réduction du temps d'onboarding des nouveaux développeurs de 500\%
            \item Passage du setup projet de 1 semaine (en moyenne) à 1 journée
            \item Passage d'un déploiement manuel à un déploiement automatisé
            \item Réduction du downtime
            \item Intégration de Docker dans le pipeline CI/CD
            \item Tests automatisés de bout en bout
        \end{itemize}
        \item Participation à l'amélioration globale de la sécurité des données
    \end{itemize}}

\cventry{2023}{Ingénieur Logiciel}{LA POSTE}{Freelance}{Télétravail}{
    \begin{itemize}
        \item Développement d'une plateforme IaaS offrant des services cloud internes.
        \item Création d'un nouveau frontend en Angular.
        \item Gestion des machines virtuelles, de la sécurité et des certificats.
        \item Intégration de nouvelles fonctionnalités
        \item Optimisation des fonctionnalités existantes.
        \item Otpimisations des requêtes SQL et des performances de l'application (durées de chargements divisés par 15 en moyenne)
    \end{itemize}}

\cventry{2022}{Tech-Lead}{LENGOW}{Freelance}{Télétravail}{
    \begin{itemize}
        \item Mise à niveau de la base technique avec une mise à jour significative de Django.
        \item Optimisation des performances de l'application et réduction de la dette technique.
        \item Passage d'une architecture synchronisée à une architecture asynchrone.
        \item Résolution des défis techniques en travaillant en étroite collaboration avec les équipes de développement.
    \end{itemize}}

\cventry{2022}{Ingénieur Logiciel}{SIGNAUX FAIBLES}{Freelance}{Télétravail}{
    \begin{itemize}
        \item Amélioration des performances de la base de données (de 30s à 1s de temps de requête)
        \item Développement de nouvelles fonctionnalités backend en Golang
        \item Optimisation des scripts d'importation de données en TypeScript
        \item Mise en œuvre d'algorithmes de détection de fraude
        \item Environnement financier complexe avec de nombreuses contraintes (zone blanche)
    \end{itemize}}

\cventry{2021}{Architecte Logiciel}{DEEPREACH}{Freelance}{Télétravail}{
    \begin{itemize}
        \item Architecture du backend et du frontend de l'application
        \item Réduction du temps de réponse actuel de 50\% (de 2s à 1s)
        \item Passage d'une architecture monolithique à une architecture microservices
        \item Ajout de VueJS au frontend
        \item Développement de nouvelles fonctionnalités et optimisation des fonctionnalités existantes
        \item Assurer la qualité du code et le respect des meilleures pratiques
        \item Environnement AWS avec de nombreux appels externes
        \item Optimisation et surveillance des coûts AWS
    \end{itemize}}

\cventry{2021}{Tech-Lead data}{CHANEL}{Freelance}{Télétravail}{
    \begin{itemize}
        \item Architecture et développement de POC orientés IA utilisant des microservices
        \item Collaboration avec le laboratoire d'innovation de CHANEL basé à New York
        \item Traitement des données vocales et intégration des services Azure
        \item Mise en place d'une pipeline CI/CD
        \item Services écrits en Python (FastAPI), client web en React et client de bureau en C\#
    \end{itemize}}

\cventry{2019}{Tech-Lead}{OPEAZ}{Freelance}{Télétravail}{
    \begin{itemize}
        \item Création et gestion d'une équipe de 5 développeurs dans différents pays (Israël, France \& Brésil)
        \item Guide de l'équipe dans les choix techniques
        \item Optimisation du code et révision des pull requests
        \item Passage d'une architecture monolithique à une architecture microservices (Django)
        \item Développement de traitement de données (scraping, APIs) et gestion de projets utilisant Django.
        \item Mise en place d'une pipeline CI/CD et d'outils de surveillance
        \item Optimisation des coûts réduisant les factures de 30\%
        \item Assurer la qualité du code et le respect des meilleures pratiques
        \item Assurer la surveillance technique de la plateforme et la résolution des incidents
        \item Développement de l'utilisation de Docker comme outil de développement et de déploiement
        \item Aider l'entreprise à lever son premier tour de financement (levée réussie de 1M€)
    \end{itemize}}

\cventry{2019}{Tech-Lead data}{RATP}{Freelance}{Télétravail}{
    \begin{itemize}
        \item Diriger le développement de projets axés sur les données
        \item Développement de proof-of-concepts
        \item Création de tableaux de bord pour surveiller la consommation des API 
        \item Création d'un pipeline pour le traitement de données massives
    \end{itemize}}

\cventry{2017}{Lead Data Scientist}{CAISSE D’EPARGNE NORMANDIE}{Télétravail}{}{
    \begin{itemize}
        \item Premier data scientist de l'entreprise
        \item Promouvoir la culture de la “donnée” au sein de l'entreprise en animant des ateliers
        \item Développement de cas d'utilisation de l'apprentissage automatique
        \item Création d'un algorithme OCR primé
        \item Démonstration de résultats tangibles dans l'automatisation des processus
        \item Contribuer au recrutement de nouveaux data scientists
        \item Collaboration avec BPCE (groupe principal) sur des projets de données
    \end{itemize}}

\cventry{2016}{Data Scientist}{MATMUT}{Contrat à Durée Indéterminée}{Sur Site}{
    \begin{itemize}
        \item Participation à divers projets allant du développement web à la création de tableaux de bord pour de grandes entreprises comme Total et Matmut.
        \item Développement de modèles prédictifs pour l'analyse des données clients et la visualisation des résultats.
    \end{itemize}}

\cventry{2014}{Développeur Web Fullstack}{Junior Entreprise INSA Rouen (AJIR)}{Rouen}{}{
    \begin{itemize}
        \item Développement de projets web pour des clients, y compris une plateforme de gestion d'audit pour Bouygues Construction.
    \end{itemize}}

% Éducation
\section{Éducation}
\cventry{2013}{Ingénierie Médicale}{Licence}{Université des Sciences}{Rouen}{Spécialisation en systèmes de soutien respiratoire et imagerie médicale.}
\cventry{2016}{Ingénierie en Architecture Logicielle}{Master}{INSA}{Rouen}{Spécialisé en systèmes d'information et développement logiciel.}

% Projets Personnels
\section{Projets Personnels}
\cventry{2022}{Bibliochouette 2.0}{En cours d'exploitation}{}{}{
    \begin{itemize}
        \item Application web actuellement en développement (Django / React / AWS), avec une sortie prévue pour début 2024.
    \end{itemize}}

\cventry{2018}{Bibliochouette : Application Mobile}{Migré}{}{}{
    \begin{itemize}
        \item Application de gestion de bibliothèque disponible sur Android et iOS, avec environ 2000 utilisateurs en décembre 2022.
    \end{itemize}}

\cventry{2020}{Geochat.fr : Application Web}{Abandonné}{}{}{
    \begin{itemize}
        \item Créée pour faciliter la communication pendant le confinement COVID, permettant des salons de discussion basés sur la localisation.
    \end{itemize}}

\end{document}