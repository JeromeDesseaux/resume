\documentclass[11pt,a4paper,sans]{moderncv}
\moderncvstyle{classic} % Options : 'casual', 'classic', 'banking'
\moderncvcolor{blue}    % Options : 'blue', 'orange', 'green', 'red', 'purple'

\usepackage[utf8]{inputenc}
\usepackage[scale=0.82]{geometry}
\usepackage{graphicx}   % Requis pour inclure des images
\setlength{\hintscolumnwidth}{2.5cm}

% Informations personnelles
\name{Jérôme}{DESSEAUX}
\title{\textnormal{\textls[150]{Ingénieur Logiciel Senior \& Tech Lead}}}
\address{29 Allée du Petit Houx}{76230, Bois Guillaume, France}
\phone[mobile]{(+33) (0)6 83 10 42 78}
\email{jerome.desseaux@skuld.fr}
\social[github]{JeromeDesseaux}
\quote{\textit{\textcolor{darkgray}{Tech Lead et Ingénieur Logiciel Senior spécialisé dans la construction et le pilotage d'équipes à travers des systèmes backend haute performance. Combinant une expertise technique Python hands-on en architecture distribuée avec un leadership éprouvé dans le mentorat d'ingénieurs et la conduite de décisions architecturales pour des applications critiques.}}}
% Ajouter votre photo
\photo[60pt][0.8pt]{moi.jpg}

\begin{document}

% En-tête
\makecvtitle

% Résumé
\section{Résumé}
Tech Lead et Ingénieur Logiciel Senior chevronné avec plus de \textit{10 ans d'expérience} dans la construction de systèmes backend haute performance. Spécialisé dans la traduction des besoins métier complexes en stratégies architecturales concrètes et le pilotage de la mise en œuvre technique du concept à la production.

\vspace{0.1cm}

Expertise approfondie sur la pile Python \textbf{(FastAPI, Django, Flask)} et les architectures distribuées (microservices, patterns asynchrones, RabbitMQ, architecture hexagonale). Historique prouvé en matière de :

\begin{itemize}
    \item \textbf{Conduite de migrations majeures :} Transitions monolithe vers microservices/FastAPI avec déploiements sans interruption.
    \item \textbf{Excellence en performance :} Gestion de systèmes critiques traitant 1M de requêtes/heure ; réalisation d'améliorations allant jusqu'à 50\% de réduction des temps de réponse et optimisations de requêtes SQL de facteur 30x.
    \item \textbf{Leadership \& Excellence d'Ingénierie :} Établissement de standards de qualité (tests, revues de code, CI/CD), mentorat d'équipes sur les meilleures pratiques, et direction d'équipes distribuées (jusqu'à 5 ingénieurs dans plusieurs pays).
\end{itemize}

% Compétences Techniques
\section{Compétences Techniques}
\subsection{Expertise Principale}
\cvitem{Pile Python}{\textbf{FastAPI, Django, Flask} -- 10+ ans de développement de systèmes backend haute performance}
\cvitem{Architecture Backend}{Conception microservices, développement d'API RESTful, patterns async/await, décomposition de services}
\cvitem{Optimisation de Performance}{Optimisation de requêtes, stratégies de cache, profilage système, améliorations de scalabilité}
\cvitem{Leadership Technique}{Gestion d'équipes distribuées, décisions d'architecture, revue de code, mentorat}

\subsection{Technologies Backend}
\cvitem{Langages Principaux}{Python (FastAPI, Django, Flask, async/await, Pydantic, SQLAlchemy)}
\cvitem{Langages Secondaires}{Go, TypeScript}

\subsection{Frontend}
\cvitem{Langages}{JavaScript, TypeScript}
\cvitem{Frameworks}{React.js, Angular.js, Vue.js}

\subsection{DevOps \& Infrastructure}
\cvitem{Plateformes Cloud}{AWS, GCP}
\cvitem{Conteneurisation}{Docker, Kubernetes}
\cvitem{CI/CD}{Tests automatisés, pipelines de déploiement, monitoring}
\cvitem{Versioning}{Git, GitHub, GitLab}

\subsection{Données \& Bases de Données}
\cvitem{Bases de Données}{Optimisation SQL, tuning de performances}
\cvitem{Traitement de Données}{Pipelines ETL, web scraping (Beautiful Soup), analyse de données}

\subsection{Méthodologies}
\cvitem{Agile}{SCRUM, revues de code, amélioration continue}

% Expérience Professionnelle
\section{Expérience Professionnelle}

\cventry{Actuellement}{Ingénieur Python Senior \& Tech Lead}{BackMarket}{Freelance}{Télétravail}{
    \begin{itemize}
        \item Dirigé l'extraction architecturale du domaine checkout depuis le \textbf{monolithe Django vers un microservice FastAPI}, guidant l'équipe sur les principes d'\textbf{architecture hexagonale} pour assurer une séparation claire des préoccupations et un code maintenable à l'échelle.
        \item Conçu et implémenté un système de traitement de commandes haute performance gérant \textbf{1 million de requêtes par heure}, coachant les ingénieurs sur les patterns scalables pour les flux de checkout e-commerce critiques axés sur la fiabilité et la performance.
        \item Intégré les \textbf{files de lettres mortes RabbitMQ} pour une gestion d'erreurs résiliente et une récupération d'échecs complète, mentorant l'équipe sur les patterns d'architecture distribuée pour la fiabilité système.
        \item Promu la qualité logicielle en établissant des standards de test et en dirigeant les revues de code techniques, assurant un code prêt pour la production et les meilleures pratiques d'ingénierie tout au long de la migration.
        \item Piloté des initiatives d'excellence technique incluant les mises à niveau Python et Django, le monitoring de performance, et les améliorations de sécurité tout en mentorant les ingénieurs sur les patterns architecturaux avancés.
    \end{itemize}}

\cventry{2024-2025}{Ingénieur Python Senior \& Tech Lead}{Lengow}{Freelance}{Télétravail}{
    \begin{itemize}
        \item Dirigé l'équipe technique sur une plateforme d'intégration e-commerce construite avec \textbf{Python/Django}, pilotant les décisions d'architecture et assurant la qualité du code à travers le cycle de développement tout en mentorant les ingénieurs sur les meilleures pratiques.
        \item Architecturé et implémenté un nouveau connecteur marketplace (Octopia \& TikTokShop), concevant des patterns d'intégration scalables pour la plateforme Lengow servant plusieurs canaux e-commerce.
        \item Orchestré des \textbf{mises à niveau majeures Python et Django} à travers les services de production, assurant la compatibilité ascendante et les déploiements sans interruption.
        \item Conçu et implémenté une stratégie complète de développement et déploiement basée sur Docker :
        \begin{itemize}
            \item Réduit le temps d'onboarding des développeurs de 500\% (d'une semaine à un jour).
            \item Architecturé une infrastructure microservices conteneurisée permettant les déploiements automatisés.
            \item Intégré Docker dans le pipeline CI/CD avec tests automatisés de bout en bout.
            \item Minimisé le downtime en production grâce à l'orchestration de conteneurs.
        \end{itemize}
        \item Mentoré l'équipe de développement sur les meilleures pratiques Python, les standards de revue de code, et les patterns architecturaux.
        \item Dirigé le monitoring des performances de la plateforme, la résolution d'incidents techniques, et les initiatives d'amélioration de la sécurité.
    \end{itemize}}

\cventry{2023}{Ingénieur Python Senior}{La Poste}{Freelance}{Télétravail}{
    \begin{itemize}
        \item Développé une plateforme IaaS fournissant des services cloud internes, construisant un frontend Angular pour le provisionnement de machines virtuelles, la gestion de la sécurité et l'administration des certificats.
        \item Dirigé l'optimisation des requêtes SQL et les améliorations de performances applicatives, atteignant une réduction de 15x des temps de chargement moyens grâce au tuning de base de données et aux stratégies de cache.
        \item Intégré de nouvelles fonctionnalités de plateforme et optimisé les fonctionnalités existantes, améliorant les performances système globales et l'expérience utilisateur.
    \end{itemize}}

\cventry{2022}{Ingénieur Python Senior}{Lengow}{Freelance}{Télétravail}{
    \begin{itemize}
        \item Dirigé une mise à niveau majeure du framework Django, modernisant la base technique et réduisant la dette technique à travers le codebase.
        \item Architecturé la transition de patterns de traitement synchrones vers asynchrones utilisant async/await, améliorant les performances et la scalabilité de l'application.
        \item Collaboré avec les équipes de développement sur des défis techniques complexes, fournissant guidance et mentorat tout au long du processus de mise à niveau.
    \end{itemize}}

\cventry{2022}{Ingénieur Python Senior}{BetaGouv -- Signaux Faibles}{Freelance}{Télétravail}{
    \begin{itemize}
        \item Optimisé les performances de base de données via le tuning de requêtes et les stratégies d'indexation, réduisant les temps de requête de 30s à 1s (amélioration de facteur 30x).
        \item Développé de nouvelles fonctionnalités backend en Golang et optimisé les scripts d'importation de données en TypeScript pour une efficacité de traitement améliorée.
        \item Implémenté des algorithmes de détection de fraude pour le monitoring de transactions financières dans un environnement hautement régulé.
        \item Livré des solutions dans le cadre de contraintes de sécurité strictes et de zones opérationnelles restreintes typiques des systèmes financiers complexes.
    \end{itemize}}

\cventry{2021}{Ingénieur Python Senior \& Tech Lead}{DeepReach}{Freelance}{Télétravail}{
    \begin{itemize}
        \item Dirigé la refonte complète du système backend et frontend, définissant les frontières de services et les contrats d'API pour guider l'équipe à travers la transition microservices.
        \item Mené l'évolution architecturale du monolithique vers l'architecture microservices utilisant Python, atteignant 50\% de réduction du temps de réponse (de 2s à 1s).
        \item Guidé l'équipe dans la conception d'une architecture backend scalable optimisée pour l'environnement cloud AWS avec multiples intégrations d'API tierces.
        \item Implémenté des stratégies d'optimisation de performance incluant le cache, l'optimisation de requêtes, et les patterns de traitement asynchrone.
        \item Intégré un framework frontend moderne (VueJS) avec des APIs backend RESTful, assurant une séparation claire des préoccupations.
        \item Mentoré l'équipe de développement en établissant des standards de qualité de code, des patterns architecturaux, et des meilleures pratiques à travers l'organisation.
        \item Optimisé les coûts d'infrastructure AWS via le monitoring des ressources, le dimensionnement optimal, et la conception de services efficiente.
    \end{itemize}}

\cventry{2021}{Tech-Lead Data}{Chanel}{Freelance}{Télétravail}{
    \begin{itemize}
        \item Dirigé l'architecture et le développement d'une plateforme IA basée microservices utilisant \textbf{Python FastAPI} pour les services backend, intégrant le traitement de données vocales avec les services cloud Azure.
        \item Conçu une architecture API RESTful avec FastAPI exploitant les patterns async/await, les modèles Pydantic pour la validation de données, et la documentation OpenAPI pour une intégration frontend fluide.
        \item Construit des services backend scalables gérant le traitement de données vocales en temps réel, coordonnant entre plusieurs microservices pour l'inférence de modèles IA et la gestion de pipelines de données.
        \item Livré une solution full-stack : backend FastAPI, client web React, et application desktop C\#, en collaboration avec le laboratoire d'innovation de CHANEL basé à New York.
        \item Établi un pipeline CI/CD et des pratiques DevOps pour les tests automatisés et le déploiement de l'architecture microservices.
        \item Dirigé les décisions d'architecture technique et le développement de POC, démontrant la viabilité en production de l'approche microservices pour les applications IA.
    \end{itemize}}

\cventry{2019}{Ingénieur Python Senior \& Tech Lead}{Opeaz}{Freelance}{Télétravail}{
    \begin{itemize}
        \item Construit et dirigé une équipe de développement internationale de cinq ingénieurs à travers Israël, France et Brésil, gérant la collaboration distribuée et le mentorat technique.
        \item Architecturé et exécuté la transition du monolithique vers l'\textbf{architecture microservices basée Django}, définissant les frontières de services et la stratégie de migration.
        \item Conçu et développé des pipelines de traitement de données utilisant Python (Django) pour le web scraping, les intégrations d'API, et les workflows ETL.
        \item Établi des pratiques d'excellence en ingénierie : revues de code complètes, pipelines CI/CD, infrastructure de monitoring, et workflows de déploiement basés Docker.
        \item Piloté les décisions d'architecture technique et guidé l'équipe à travers les défis complexes de développement backend.
        \item Réduit les coûts opérationnels de 30\% via l'optimisation d'infrastructure et la conception système efficiente.
        \item Contribué au succès business en livrant une fondation technique robuste, supportant le premier tour de financement de l'entreprise (1M€ levé).
    \end{itemize}}

\cventry{2019}{Tech-Lead Data}{RATP}{Freelance}{Télétravail}{
    \begin{itemize}
        \item Dirigé le développement de projets orientés données et de proof-of-concepts pour l'analytique et l'optimisation des opérations de transport.
        \item Construit des pipelines de traitement de données en Python pour gérer des datasets de transport à grande échelle, permettant l'analyse et le reporting en temps réel.
        \item Créé des tableaux de bord de monitoring pour le suivi de consommation d'API, l'analyse de performance système, et la visualisation de métriques opérationnelles.
    \end{itemize}}

\cventry{2017}{Lead Data Scientist}{Caisse d'Epargne Normandie}{Télétravail}{}{
    \begin{itemize}
        \item Pionnier de la fonction data science en tant que data scientist fondateur de l'entreprise, établissant la fondation technique pour les initiatives ML.
        \item Promu une culture data-driven en organisant des ateliers et sessions de formation pour les parties prenantes à travers l'organisation.
        \item Développé des cas d'usage de machine learning et créé un algorithme OCR primé pour l'automatisation du traitement de documents.
        \item Démontré des résultats tangibles dans l'automatisation des processus, contribuant au recrutement de data scientists additionnels lors de la croissance de l'équipe.
        \item Collaboré avec BPCE (groupe parent) sur des projets data stratégiques et le partage de meilleures pratiques inter-organisations.
    \end{itemize}}

\cventry{2016}{Data Scientist}{Matmut}{Contrat à Durée Indéterminée}{Sur Site}{
    \begin{itemize}
        \item Développé des modèles prédictifs et des solutions de visualisation de données pour clients entreprise.
    \end{itemize}}

\cventry{2015}{Développeur C\# .NET}{Devolis}{Contrat à Durée Indéterminée}{Sur Site}{
    \begin{itemize}
        \item Construit des applications web et des APIs RESTful utilisant ASP.NET pour clients entreprise (Total, Matmut).
        \item Déployé des applications sur Microsoft Azure avec backend SQL Server.
    \end{itemize}}

\cventry{2014}{Développeur Web Fullstack}{Junior Entreprise INSA Rouen (AJIR)}{Rouen}{}{
    \begin{itemize}
        \item Développé des applications web pour clients incluant Bouygues Construction.
    \end{itemize}}

% Éducation
\section{Éducation}
\cventry{2013}{Ingénierie Médicale}{Licence}{Université des Sciences}{Rouen}{Spécialisé en systèmes de support respiratoire et imagerie médicale.}
\cventry{2016}{Ingénierie en Architecture Logicielle}{Master}{INSA}{Rouen}{Spécialisé en systèmes d'information et développement logiciel.}

% Projets Personnels
\section{Projets Personnels}
\cventry{2022}{Bibliochouette PWA}{Production}{\href{https://app.bibliochouette.fr}{app.bibliochouette.fr}}{}{
    \begin{itemize}
        \item Réécriture complète de l'application mobile Bibliochouette utilisant backend Go, frontend React PWA, et base de données PostgreSQL.
        \item Amélioré les performances, la scalabilité, et l'expérience utilisateur comparé à l'application mobile originale.
        \item Implémenté de nouvelles fonctionnalités incluant recherche améliorée, fonctionnalités sociales, et responsive mobile amélioré.
        \item Discussions actuellement en cours avec des associations pour un déploiement social à travers le monde.
    \end{itemize}}

\cventry{2018}{Bibliochouette : Application Mobile}{Déprécié}{}{}{
    \begin{itemize}
        \item Application de gestion de bibliothèque disponible sur Android et iOS, avec environ 2 000 utilisateurs en décembre 2022.
        \item Déprécié en faveur d'un projet web pour une meilleure maintenabilité et compatibilité cross-platform.
        \item Utilisation de Firebase pour les services backend \& Flutter pour le développement frontend.
    \end{itemize}}

\cventry{2020}{Geochat.fr : Application Web}{Abandonné}{}{}{
    \begin{itemize}
        \item Créé pour faciliter la communication pendant les confinements COVID, permettant des salles de chat basées sur la localisation.
    \end{itemize}}

% Hobbies
\section{Hobbies}
\cventry{}{Course à pied}{Records Personnels}{}{}{
    \begin{itemize}
        \item Semi-marathon (Bois Guillaume 2025) : 01:56:18
        \item 10km (Rouen 2025) : 00:48:59
    \end{itemize}}

\cventry{}{Lecture}{}{}{}{
    \begin{itemize}
        \item \textit{Clean Architecture: A Craftsman's Guide to Software Structure and Design} – Robert C. Martin
        \item \textit{The Pragmatic Programmer: Your Journey to Mastery} – Andrew Hunt et David Thomas
        \item \textit{Cloud Design Patterns: Prescriptive Architecture Guidance for Cloud Applications} – Microsoft Press
        \item \textit{Designing Data-Intensive Applications} – Martin Kleppmann
        \item \textit{Soft Skills: The Software Developer's Life Manual} – John Sonmez
        \item \textit{Azure for Architects: Implementing Cloud Design, DevOps, and Microservices} – Ritesh Modi
    \end{itemize}
}

\end{document}
